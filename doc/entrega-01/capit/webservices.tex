
\section{Web Services}
\label{sec:web_services}

Nesta seção será descrita toda a {\it API} utilizada para acessar a aplicação
através de {\it Web Services}. A seção \ref{sec:premissas} apresenta as 
premissas que devem ser levadas e a \ref{sec:convencoes} as convenções em conta ao se analisar a {\it API}.  

 
\subsection{Premissas}
\label{sec:premissas}

\begin{enumerate}
    \item Todas os textos entre [] são variáveis;
    \item Todos os retornos são JSON;
    \item O retorno default é um array associativo que contém um código de 
            erro e sua respectiva mensagem. Se o código de erro for 0 (zero)
            quer dizer que não forão encontrados erros;
    \item Os argumentos iniciados com * são opcionais.
\end{enumerate}
 
\subsection{Convenções}
\label{sec:convencoes}

Nesta seção serão utilizados os seguintes conceitos:

\begin{itemize}
    \item Método
    \item Objeto instanciado
    \item Coleção (lista de objetos instanciados)
\end{itemize}

Um texto seguido por / (barra) em uma URL representa membros de um objeto 
instanciado, e os seguidos por . (ponto) representam membros de uma coleção. 
Por exemplo:

O método {\bf modify} é parte de um objeto instanciado do tipo scanner:
\begin{verbatim}
    /scanner/[name]/modify
\end{verbatim}

O método {\bf add} é parde de uma coleção de scanners:
\begin{verbatim}
    /scanner.add
\end{verbatim}

\subsection{Serviços disponíveis}
\label{sec:servicos_disponiveis}

\subsubsection{Scanner}

\begin{itemize}

\item Mostrar os {\it scanners} configurados:
\begin{verbatim}
URL: /scanner
Argumentos: Nenhum
Método: GET, POST
Retorna: Collection containing all scanner objects.
\end{verbatim}

\item Adicionar um novo {\it scanners}:
\begin{verbatim}
URL: /scanner.add
Argumentos: name, localization*, description*, manufacturer, 
    model, papersize*, isindexable*, user*, group*
Método: POST
Retorna: Scanner object.
\end{verbatim}

\item Retornar um objeto {\it scanner}:
\begin{verbatim}
URL: /scanner/[name]
Argumentos: Nenhum
Método: GET, POST
Retorna: Scanner object.
\end{verbatim}

\item Modificar a configuração de um {\it scanner}:
\begin{verbatim}
URL: /scanner/[name]/modify
Argumentos: name*, localization*, description*, manufacturer*, 
    model*, papersize*, isindexable*, user, group
Método: POST
Retorna: Scanner object.
\end{verbatim}

\item Remover um {\it scanner}:
\begin{verbatim}
URL: /scanner/[name]/delete
Argumentos: Nenhum
Método: DELETE
Retorna: Default
\end{verbatim}

\item Retornar o código de status do {\it scanner}:
\begin{verbatim}
URL: /scanner/[name]/status
Argumentos: Nenhum
Método: GET, POST
Retorna: Default
\end{verbatim}

\item Digitalizar uma página:
\begin{verbatim}
URL: /scanner/[name]/scan
Argumentos: Nenhum
Método: GET, POST
Retorna: Page object.
\end{verbatim}

\end{itemize}

\subsubsection{User}

\begin{itemize}

\item Listar todos os usuários registrados:
\begin{verbatim}
URL: /user
Argumentos: Nenhum
Método: GET, POST
Retorna: Collection containing all user objects.
\end{verbatim}

\item Adicionar um novo usuário:
\begin{verbatim}
URL: /user.add
Argumentos: username, password, role
Método: POST
Retorna: User object.
\end{verbatim}

\item Retornar um objeto usuário:
\begin{verbatim}
URL: /user/[username]
Argumentos: Nenhum
Método: GET, POST
Retorna: User object.
\end{verbatim}

\item Modificar o registro de um usuário:
\begin{verbatim}
URL: /user/[username]/modify
Argumentos: username*, password*, role*
Método: POST
Retorna: User object.
\end{verbatim}

\item Deletar um usuário:
\begin{verbatim}
URL: /user/[username]/delete
Argumentos: Nenhum
Método: DELETE
Retorna: Default
\end{verbatim}

\item Listar de todos as páginas digitalizadas e não utilizadas: 
\begin{verbatim}
URL: /user/[username]/pages
Argumentos: Nenhum
Método: GET, POST
Retorna: Collection of page objects.
\end{verbatim}

\item Listar os documentos gerados:
\begin{verbatim}
URL: /user/[username]/document
Argumentos: Nenhum
Método: GET, POST
Retorna: Collection of document objects.
\end{verbatim}

\item Criar um novo documento usando as páginas selecionadas:
\begin{verbatim}
URL: /user/[username]/document.create
Argumentos: docname, docdescript, pages (list of page ids)
Método: GET, POST
Retorna: Document object.
\end{verbatim}

\item Baixar um documento digitalizado:
\begin{verbatim}
URL: /user/[username]/document/[docname]
Argumentos: Nenhum
Método: GET, POST
Retorna: Downloadable PDF document.
\end{verbatim}

\end{itemize}

\subsubsection{Auth}

\begin{itemize}

\item Autenticar um usuário:
\begin{verbatim}
URL: /auth/signin
Argumentos: username, password
Método: POST
Retorna: Default
\end{verbatim}

\item Desautenticar um usuário:
\begin{verbatim}
URL: /auth/signout
Argumentos: Nenhum
Método: GET, POST
Retorna: Default
\end{verbatim} 

\end{itemize}
