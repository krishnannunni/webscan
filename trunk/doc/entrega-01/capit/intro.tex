\section{Introdução}
\label{sec:intro}

Durante a primeira fase da elaboração do projeto {\it webscan}, foram feitas as seguintes atividades:

\begin{description}
    \item[Casos de uso: ] Nessa atividade, descrita na seção \ref{sec:casos_de_uso}, foram elaborados diagramas de casos de uso, representando as principais interações entre atores e o sistema. Também foram elaborados os casos de uso completo-abstrato, de forma a dar detalhamento a cada caso de uso contido no diagrama;
    \item[Mockups e storyboards: ] Para essa atividade, descrita com mais detalhes na seção \ref{sec:mockups}, foram elaboradas as candidatas às telas de interface do sistema ({\it mockups} e também o curso de ações que um usuário pode realizar {\it storyboards};
    \item[Pesquisa de bibliotecas: ] Para o desenvolvimento do projeto, essa atividade (descrita com mais detalhes na seção \ref{sec:pesquisa_libs}) consistiu na elaboração de uma pesquisa sobre as bibliotecas de desenvolvimento para a digitalização do documento;
\end{description}

\subsection{Terminologia}

\subsubsection{Atividade de Desenvolvimento:}
Atividade de desenvolvimento se refere à quantidade de escritas (ou seja, código sendo atualizado/adicionado) em um sistema de controle de versões, quando disponível.

\begin{description}
    \item[Alta:] diversas atividades no último mês.
    \item[Baixa:] algumas atividades ao longo dos últimos 3 meses.
    \item[Parado:] não houve nenhuma atividade de escrita nos últimos 6 meses.
\end{description}

