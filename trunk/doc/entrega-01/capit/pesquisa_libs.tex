\section{Bibliotecas de digitalização}
\label{sec:pesquisa_libs}

Uma das atividades na fase atual do projeto foi elaborar uma pesquisa sobre as principais bibliotecas de digitalização de documentos disponíveis para plataformas Microsoft Windows, através da interface TWAIN e para plataforma GNU/Linux, através da interface SANE. 

Na seção \ref{sec:twain}, tem-se as principais bibliotecas encontradas para o uso da interface TWAIN e então, na seção \ref{sec:sane}, tem-se as principais bibliotecas encontradas para o uso da interface SANE.

\subsection{TWAIN}
\label{sec:twain}

\subsubsection{Descrição}
TWAIN é a interface de câmeras digitais e scanners específica para plataforma Windows 32 bits apenas. Não suporta scanners distribuídos na rede e não separa interface do driver (segundo www.sane-project.org).

\subsubsection{Bibliotecas}
\begin{description*}
	\item[Nome:] TWAIN Module
	\item[Linguagem(ns):] Python (versões 2.1 até 2.5)
	\item[Licença:] GPLv2
	\item[Plataforma(s):] Windows 32 bits
	\item[Endereço:] http://twainmodule.sourceforge.net
	\item[Última versão:] 1.0.3
	\item[Data da última atualização do site:] Novembro de 2006
	\item[Data do último {\it release}:] 31 de maio de 2007
	\item[Atividade de desenvolvimento:] parado
	\item[Descrição:] 
	Bem completa. Suporta ativ idades básicas TWAIN como funções prontas ou funções TWAIN avançadas que podem ser implementadas. Possui documentação ampla, com tutoriais, referências, guias e exemplos.
\end{description*}

\begin{description*}
	\item[Nome:] EZTwain
	\item[Linguagem(ns):] C, C++, Visual Basic, Delphi, entre outros
	\item[Licença:] Domínio público
	\item[Plataforma(s):] Windows 32 bits
	\item[Endereço:] http://www.dosadi.com/eztwain1.htm
	\item[Última versão:] 1.16
	\item[Data da última atualização do site:] Não disponível
	\item[Data do último {\it release}:] 11 de maio de 2007
	\item[Atividade de desenvolvimento:] parado
	\item[Descrição:] 
	É bastante popular, inclusive é amplamente usado com um {\it wrapper} para suporte de TWAIN em Java. Documentação limitada, porém possui exemplos prontos em C. Para C, foi testado apenas em Visual C++ 5 e 6.
\end{description*}

\begin{description*}
	\item[Nome:] TWAIN Development Kit
	\item[Linguagem(ns):] C, C++ (Visual Studio)
	\item[Licença:] Privada
	\item[Plataforma(s):] Windows 32 bits
	\item[Endereço:] http://www.twain.org
	\item[Última versão:] Não disponível
	\item[Data da última atualização do site:] Não disponível
	\item[Data do último {\it release}:] Não disponível
	\item[Atividade de desenvolvimento:] Não disponível
	\item[Descrição:] 
	Documentação esparsa, falta exemplos
\end{description*}

%%%%%%%%%%%%%%%%%%%%%%%%%%%%%%%%%%%%%%%%%%%%%%%%%%%%%%%%%%%%%%%%%%%%%%%%%%%%%%%%%%%%%%%%%%%%%%%%%%%%%%%%%%%%%

\subsection{SANE}
\label{sec:sane}

\subsubsection{Descrição}
SANE (Scanner Access is Now Easy) é uma implementação de aquisição de imagens comum em sistemas open sources, como o Linux e FreeBSD. Há implementações para BeOS, OS/2 e MacOS X.

\subsubsection{Bibliotecas}

\begin{description*}
	\item[Nome:] SANE API
	\item[Linguagem(ns):] C
	\item[Licença:] GPL
	\item[Plataforma(s):] FreeBSD, Linux, BeOS, HP-UX, entre outros.
	\item[Endereço:] http://www.sane-project.org
	\item[Última versão:] 1.0.19
	\item[Data da última atualização do site:] Não disponível
	\item[Data do último {\it release}:] 11 de fevereiro de 2008
	\item[Atividade de desenvolvimento:] alta
	\item[Descrição:] 
	Documentação ampla, comunidade ativa, muitas implementações disponíveis para ver como exemplo.
\end{description*}

\begin{description*}
	\item[Nome:] PIL (Python Imaging Library)
	\item[Linguagem(ns):] Python (versões 2.4 e 2.5)
	\item[Licença:] ver http://www.pythonware.com/products/pil/license.htm
	\item[Plataforma(s):] FreeBSD, Linux, BeOS, HP-UX, entre outros.
	\item[Endereço:] http://www.pythonware.com/products/pil/
	\item[Última versão:] 1.1.6
	\item[Data da última atualização do site:] Não disponível
	\item[Data do último {\it release}:] 3 de dezembro de 2006
	\item[Atividade de desenvolvimento:] Não disponível
	\item[Descrição:] 
	Amplamente usada para processamento de imagens em Linux, usada por diversos front-ends que usam python. Biblioteca simples, manual com tutorial incluso no código fonte da PIL.
\end{description*}


