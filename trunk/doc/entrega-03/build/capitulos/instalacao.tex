\section{Instalação}
\label{sec:instalacao}
Este capítulo descreve os passos para a instalação do módulo daemon do
{\it webscan}. O módulo UI não exige instalação e por esse motivo não

Estas instruções estão divididas em 2 seções que descrevem os
procedimentos de instalação nos sistemas Linux e Windows respectivamente.

Em ambos sistemas a instalação dos drivers específicos dos scanners que 
serão utilizados é necessária para que software trabalhe corretamente.

\subsection{Linux}
Esta seção cobre a instalação do produto {\it webscan} na distribuição
Ubuntu e outras que compartilham o mesmo tipo de gerenciamento de pacotes.

As linhas de comando apresentadas nesta seção poderão iniciar com `\#' ou
`\$'; quando iniciadas com `\#' significa que elas precisam ser executadas 
com super-usuário (ou root), caso contrário um usuário convencional será
suficiente.

\subsubsection{Pré-Requisitos}
Boa parte das dependências do {\it webscan} podem ser instaladas através 
do software apt-get executando-se o comando:

\begin{verbatim}
 # apt-get install [nome_do_pacote nome_de_outro_pacote]
\end{verbatim}

Os pacotes necessários são:

\begin{itemize}
    \item python
    \item python-imaging
    \item python-imaging-sane
    \item python-setuptools
    \item python-simplejson
\end{itemize}

Após instalados os pacotes do sistema operacional será necessária a 
instalação do Django. Baixe a versão 1.0 em:
\begin{verbatim}
http://www.djangoproject.com/
\end{verbatim}

Após baixar o pacote do Django a instalação deve ser realizada com as 
seguintes linhas de comando:
\begin{verbatim}
 $ tar xzvf Django-1.0.tar.gz
 $ cd Django-1.0
 # python setup.py install
\end{verbatim}

\subsubsection{Módulo daemon}
Para instalar o módulo daemon será necessário a cópia do 
código-fonte para o servidor onde o scanner será utilizado.

O código-fonte pode ser obtido no CD que segue junto a este
relatório. A versão atualizada deste código também pode ser encontrada
no repositório do projeto Interlegis em:
\begin{verbatim}
http://repositorio.interlegis.gov.br/digidoc/trunk/
\end{verbatim}
 

\subsection{Windows}
